\section{Distance-based functional classification
{[}Marie{]}}\label{distance-based-functional-classification-marie}

In this section, we explore whether our geodesic distance estimator has
benefits for downstream analysis task. There are many tasks we could
consider here such as distance-based nonparametric regression and
distanced-based functional clustering, but we will focus on
distance-based functional classification. It must be noted that while
curve alignment, also known as curve registration, is necessarily
performed as a preprocessing technique prior to clustering and
classification, our geodesic distance estimator allows one to forsake
this step.

For simplicity, assume the task is binary classification. Associated to
each functional object \(x\) is a binary \(y\) indicating class
membership. Consider the classifier proposed in \cite{Ferraty2006} which
is a functional version of the Nadaraya-Watson kernel estimator of class
membership probabilities: \[
\hat p(y = 0 | x) \frac{ \sum_{i=1}^n K[h^{-1} d(x,x_i)] 1(y_i = 0) }{ \sum_{i=1}^n K[h^{-1} d(x,x_i)] }
\] We shall compare our method to using \(L^2\) distance, possibly
weighted, and with curve registration already accomplished. Describe
alternative methods in detail.

The bandwidth in the classifier should be tuned individually for each
method. Also we might need to tune MDS dimension \(s\) since in real
data, the dimension of the manifold might be much higher than
encountered in the simulation scenarios where it never goes above 2.

Datasets used by functional classification papers

\begin{itemize}
\item Wheat, rainfall and phoneme in Aurore's paper "Achieving near-perfect classification for functional data"
\item Berkeley growth curves in \cite{ChenReiss2014}.
\item Tecator and phoneme in \cite{Galeano2015} Mahalanobis technometrics paper.
\item yeast cell cycle gene expression (can't find this publicly) in \cite{LengMuller2005} "Classification using functional data analysis for temporal gene expression data"
\end{itemize}

Datasets used in functional manifold papers

\begin{itemize}
\item Berkeley growth, yeast cell cycle gene expression (can't find this publicly) in \cite{ChenMuller2012}
\item Tecator in \cite{LinYao2017} contamination paper
\item Berkeley growth, gait cycle in \cite{Dimeglio2014} robust isomap paper
\end{itemize}
